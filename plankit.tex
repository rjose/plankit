\documentclass{article}


\title{Plan Kit}
\author{Rino Jose}
\date{January 27, 2017}


\begin{document}

\maketitle

The goal is to write progams in C that can answer questions about planned work.
I'd like these programs to be pluggable into web services that can run as
little functional modules.

Queries will be sent to these modules over sockets. Responses will be returned
over sockets. There may be ways to load data outside of this mechanism, but
it should also be possible to create and modify records in this way.

There should be a set of programs that can answer questions non-destructively.
That is, they should be able to operate on a version of data and create
alternate versions for use in scenario-based analysis. If possible, this
alternate reality tree could be represented as operations on a version of the
data and applied to the data at different points. Perhaps a DSL would be appropriate
to describe these types of changes.

Along these lines, it would be useful if we could create optimizers that could
explore various alternate realities to identify interesting cases to consider.

It would be helpful to be able to describe the observed reality in the same terms
so that one could generate alternate realities by changing the base version while
maintaining actual events constant.

\section{Steps}
Here are some steps to get started

\begin{enumerate}
    \item Get infrastructure in place to build a basic app

    \item Create a model/DSL for modeling state and events
\end{enumerate}

The first step is straightforward. We'll follow along with the ``21st Century C'' book.


\end{document} 
